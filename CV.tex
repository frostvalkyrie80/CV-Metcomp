%%%%%%%%%%%%%%%%%
% This is an sample CV template created using altacv.cls
% (v1.7, 9 August 2023) written by LianTze Lim (liantze@gmail.com). Compiles with pdfLaTeX, XeLaTeX and LuaLaTeX.
%
%% It may be distributed and/or modified under the
%% conditions of the LaTeX Project Public License, either version 1.3
%% of this license or (at your option) any later version.
%% The latest version of this license is in
%%    http://www.latex-project.org/lppl.txt
%% and version 1.3 or later is part of all distributions of LaTeX
%% version 2003/12/01 or later.
%%%%%%%%%%%%%%%%

%% Use the "normalphoto" option if you want a normal photo instead of cropped to a circle
% \documentclass[10pt,a4paper,normalphoto]{altacv}

\documentclass[10pt,a4paper,ragged2e,withhyper]{altacv}
%% AltaCV uses the fontawesome5 and packages.
%% See http://texdoc.net/pkg/fontawesome5 for full list of symbols.

% Change the page layout if you need to
\geometry{left=1.25cm,right=1.25cm,top=1.5cm,bottom=1.5cm,columnsep=1.2cm}

% The paracol package lets you typeset columns of text in parallel
\usepackage{paracol}

% Change the font if you want to, depending on whether
% you're using pdflatex or xelatex/lualatex
% WHEN COMPILING WITH XELATEX PLEASE USE
% xelatex -shell-escape -output-driver="xdvipdfmx -z 0" sample.tex
\ifxetexorluatex
  % If using xelatex or lualatex:
  \setmainfont{Roboto Slab}
  \setsansfont{Lato}
  \renewcommand{\familydefault}{\sfdefault}
\else
  % If using pdflatex:
  \usepackage[rm]{roboto}
  \usepackage[defaultsans]{lato}
  % \usepackage{sourcesanspro}
  \renewcommand{\familydefault}{\sfdefault}
\fi

% Change the colours if you want to
\definecolor{SlateGrey}{HTML}{2E2E2E}
\definecolor{LightGrey}{HTML}{666666}
\definecolor{DarkPastelRed}{HTML}{450808}
\definecolor{PastelRed}{HTML}{8F0D0D}
\definecolor{GoldenEarth}{HTML}{E7D192}
\colorlet{name}{black}
\colorlet{tagline}{PastelRed}
\colorlet{heading}{DarkPastelRed}
\colorlet{headingrule}{GoldenEarth}
\colorlet{subheading}{PastelRed}
\colorlet{accent}{PastelRed}
\colorlet{emphasis}{SlateGrey}
\colorlet{body}{LightGrey}

% Change some fonts, if necessary
\renewcommand{\namefont}{\Huge\rmfamily\bfseries}
\renewcommand{\personalinfofont}{\footnotesize}
\renewcommand{\cvsectionfont}{\LARGE\rmfamily\bfseries}
\renewcommand{\cvsubsectionfont}{\large\bfseries}


% Change the bullets for itemize and rating marker
% for \cvskill if you want to
\renewcommand{\cvItemMarker}{{\small\textbullet}}
\renewcommand{\cvRatingMarker}{\faCircle}
% ...and the markers for the date/location for \cvevent
% \renewcommand{\cvDateMarker}{\faCalendar*[regular]}
% \renewcommand{\cvLocationMarker}{\faMapMarker*}


% If your CV/résumé is in a language other than English,
% then you probably want to change these so that when you
% copy-paste from the PDF or run pdftotext, the location
% and date marker icons for \cvevent will paste as correct
% translations. For example Spanish:
% \renewcommand{\locationname}{Ubicación}
% \renewcommand{\datename}{Fecha}

\begin{document}
\name{Pedro Guerreiro Aguiar Pereira}
\tagline{Astronauta Perdido}
%% You can add multiple photos on the left or right
\photoR{2.8cm}{fd.JPG}
% \photoL{2.5cm}{Yacht_High,Suitcase_High}

\personalinfo{%
  % Not all of these are required!
  \email{pedro23@ov.ufrj.br}
  \phone{21-42424242}
  \mailaddress{Pequeno buraco na esquina sudeste da Cratera Victoria, Marte}
  \location{Cratera Victoria, Marte}
  \homepage{www.flamengo.com}
  \twitter{@frostvalkyrie}
  \github{frostvalkyrie80}
  %% You can add your own arbitrary detail with
  %% \printinfo{symbol}{detail}[optional hyperlink prefix]
  % \printinfo{\faPaw}{Hey ho!}[https://example.com/]

  %% Or you can declare your own field with
  %% \NewInfoFiled{fieldname}{symbol}[optional hyperlink prefix] and use it:
  % \NewInfoField{gitlab}{\faGitlab}[https://gitlab.com/]
  % \gitlab{your_id}
  %%
  %% For services and platforms like Mastodon where there isn't a
  %% straightforward relation between the user ID/nickname and the hyperlink,
  %% you can use \printinfo directly e.g.
  % \printinfo{\faMastodon}{@username@instace}[https://instance.url/@username]
  %% But if you absolutely want to create new dedicated info fields for
  %% such platforms, then use \NewInfoField* with a star:
  % \NewInfoField*{mastodon}{\faMastodon}
  %% then you can use \mastodon, with TWO arguments where the 2nd argument is
  %% the full hyperlink.
  % \mastodon{@username@instance}{https://instance.url/@username}
}

\makecvheader
%% Depending on your tastes, you may want to make fonts of itemize environments slightly smaller
% \AtBeginEnvironment{itemize}{\small}

%% Set the left/right column width ratio to 6:4.
\columnratio{0.6}

% Start a 2-column paracol. Both the left and right columns will automatically
% break across pages if things get too long.
\begin{paracol}{2}
\cvsection{Experiências}

\cvevent{Astronauta}{NASA}{Janeiro 2082 -- ????}{Terra}
\begin{itemize}
\item Não terminou bem 
\item Vide meu título
\end{itemize}

\divider

\cvevent{Catador de pedras Marcianas}{NASA?}{???? -- Sabe se lá quando}{Marte}
\begin{itemize}
\item Horário Flexível
\item Solitário, mas relaxante
\end{itemize}

\cvsection{Objetivos}

\cvevent{Objetivo 1}{}{}{}
\begin{itemize}
\item Consertar o módulo de aceleração da nave de escape
\end{itemize}

\divider

\cvevent{Objetivo 2}{}{}{}
\begin{itemize}
\item Me adequar à cultura local
\end{itemize}
\medskip

\cvsection{Um dia comigo}

% Adapted from @Jake's answer from http://tex.stackexchange.com/a/82729/226
% \wheelchart{outer radius}{inner radius}{
% comma-separated list of value/text width/color/detail}
\wheelchart{1.5cm}{0.5cm}{%
  6/8em/accent!30/{Fugir de \\criaturas esquisitas},
  3/8em/accent!40/Trabalhar no \\conserto da nave,
  8/8em/accent!60/Procurar Alimentos,
  2/10em/accent/Dormir,
  5/6em/accent!20/Lembrar da família
}

% use ONLY \newpage if you want to force a page break for
% ONLY the current column
\newpage

\cvsection{Livros Favoritos}

\cvachievement{\faBook}{Cosmos}{Sagan, Carl. 1980}

\divider

\cvachievement{\faBook}{A morte da luz}{Martin, George R.R. 1977}

\divider

\cvachievement{\faBook}{Perdido em Marte}{Weir, Andy. 2011}

%% Switch to the right column. This will now automatically move to the second
%% page if the content is too long.
\switchcolumn

\cvsection{Frase Favorita}

\begin{quote}
``Nem todos que vagam estão perdidos.''
\end{quote}

\cvsection{Minhas conquistas}

\cvachievement{\faTrophy}{Primeiro lugar na 1ª edição da corrida maluca marciana}{Não falamos sobre a terceira volta}

\divider

\cvachievement{\faTrophy}{Prêmio Nobel de Física}{Porém esqueci em casa...}

\divider

\cvachievement{\faTrophy}{Prêmio Hughins de melhor castelo de areia vermelha}{Posso ter inventado esse}

\cvsection{Forças}

\cvtag{Perseverante}
\cvtag{Casca-Dura}\\
\cvtag{Aventureiro \& Destemido}

\divider\smallskip

\cvtag{Engenharia Aeroespacial}
\cvtag{Física Aplicada}\\
\cvtag{Eletrônica}

\cvsection{Linguas Faladas}

\cvskill{Português}{5}
\divider

\cvskill{Inglês}{4.5}
\divider

\cvskill{Marciano}{0.5} %% Supports X.5 values.

%% Yeah I didn't spend too much time making all the
%% spacing consistent... sorry. Use \smallskip, \medskip,
%% \bigskip, \vspace etc to make adjustments.
\medskip

\cvsection{Formação}

\cvevent{Doutorado em\ Engenharia Aeroespacial}{Massachusetts Institute of Technology (MIT)}{Setembro 2078 -- Junho 2080}{}
Tese de doutorado: Aplicando Inteligência Artificial nos controles de um foguete de exploração espacial

\divider

\cvevent{Mestrado em\ Engenharia Aeroespacial}{Universidade Federal do Rio de Janeiro (UFRJ)}{Sept 2075 -- June 2077}{}

\divider

\cvevent{Bacharelado em\ Astronomia}{Universidade Federal do rio de Janeiro (UFRJ)}{Janeiro 2069 -- Dezembro 2074}{}

% \divider

\cvsection{Referências}

% \cvref{name}{email}{mailing address}
\cvref{Prof.\ Albert Einstein II}{MIT}{einsteinII.albert@MIT.educ}


\divider

\cvref{Prof.\ Isaac Newton IV}{UFRJ}{newtoniv.fisica@usp.edu}



\end{paracol}


\end{document}
